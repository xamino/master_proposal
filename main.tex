\documentclass[a4paper,10pt,twoside]{article}

%clickable links
\usepackage{hyperref}
% better itemize
\usepackage{enumitem}
\usepackage{listings}
\usepackage{xcolor}
% better captions
\usepackage[font=small,labelfont=bf]{caption}
% text
\usepackage[utf8]{inputenc}
\usepackage[T1]{fontenc}
\usepackage{lmodern}
\usepackage{microtype}
% stuff
\usepackage{graphicx}
\usepackage{url}
\usepackage{color}
\usepackage[left]{eurosym}
% show author information in pdf document properties
\hypersetup{
	pdftitle={Proposal},
	pdfauthor={Tamino Hartmann}
}
% hyphenations:
\hyphenation{Ta-mi-no}
\hyphenation{Hart-mann}

\begin{document}

% define title stuff
\title{\textbf{Secure Peer to Peer File Synchronization via the Tox Protocol}\\{\Large Master Thesis Proposal}}
\author{Tamino PSM Hartmann}
\date{April 2015}
% make title
\maketitle

% no section numbering (this document is small enough not to require that)
\setcounter{secnumdepth}{0}
% define better headings (page number off to the side, etc)
\pagestyle{headings}

\section{Introduction}

The author proposes the implementation of a peer to peer file synchronization tool over the Toxcore protocol.
The goal is a secure, decentralized method of keeping files between computers synchronized without having to rely on trusted third parties.
The proposed thesis would outline the conceptual work required to design and implement a software suite that fulfills these basic goals.
This includes the specification of a protocol to facilitate the necessary communication over the data channels.

\section{Motivation}

% this paragraph might have to go, it is overly presumptuous
The thesis is mainly motivated by the revelations of drag net surveillance by Edward Snowden and the compliance by so called trusted service providers.
Thus new software solutions are required to enable users to take back their privacy on the internet without sacrificing usability and ease of access.

A range of software has come into the spotlight of public disclosure since the revelations that already work towards this goal.
One example is the Open Whisper Systems group.
Their stated goal on the website is "[...] we're working to advance the state of the art for secure communication, while simultaneously making it easy for everyone to use."~\cite{web:site:whispersystems:about}.
Another example is the Tox instant messenger community that is building a free, open source, peer to peer Skype alternative~\cite{web:site:tox}.

% move the motivation over to enabling privacy of data
The proposed thesis is to tackle the problem of cloud data storage, often used for distributed file access across multiple computers.
Existing examples of this include Google Drive~\cite{web:site:gdrive} and Dropbox~\cite{web:site:dropbox}.
While the data transfered is now encrypted in transit in most cases, it is not encrypted while residing on the servers.
This allows anyone with access to the server to fully access the data, so long as it was not already previously encrypted by the user.
Before Snowden the third parties were entrusted with the safe keeping of the stored data under the presumption that no unauthorized access was allowed or even possible.
However this trust has been lost now that the public has been made aware of the possibilities of state organizations to access any data without due legal process.
% maybe expand a bit on the trust things? --> why do we require a trusted 3rd party? what benefit does it give the user?

One solution to this problem is encrypting everything client side before uploading the data to third parties.
An example of a service that does this is Boxcryptor~\cite{web:site:boxcryptor}.
However this principle requires active involvement from the user and is thus an extra hurdle in securing one's data.
Perhaps simply forgoing a third party would be a possible solution.
Examples of existing peer to peer solutions are BitTorrent Sync~\cite{web:site:bittorrent_sync} and Syncthing~\cite{web:site:synthing}.
However now the main advantage of a third party is lost: namely the ability to store data in a way that it is readily accessible even if every peer is currently offline.
Syncthing notably explicitly states that the user can enable access for trusted third parties, but as it stands now this third party would receive an unencrypted access to any data stored there.

% importance of topic
Therefore we propose the implementation of a file synchronization program that solves these problems.
The proposed software suite should primarily utilize strong encryption to make unauthorized data access as hard as possible.
While the core features must be independent of cloud services, encrypted data storage for offline access should still be possible and allow third party services without having to entrust cleartext access.
Finally the software should be as easy to use as possible to enable anyone to use it after a simple setup process.
Once running the software should have a negligible footprint on the client's peers.

\section{Scope and Features}

In the following section we will take a look at the proposed scope and feature set of the thesis.

\subsection{Software Goals}

The stated goal is the implementation of a software set that enables the following criteria to be met.

\begin{description}[leftmargin=2em,style=nextline,noitemsep,nolistsep]
\item[Peer to peer]
	Data should be transfered directly from peer to peer, where both peers are trusted parties by the user.
	This removes the need for third party requirements such as internet access to an off site server.
	As an added bonus the software should be capable of running fully without internet access, allowing data transfers within a user's intranet to run unhindered of asymmetric internet upload speeds.
	Building on this it would be capable to use a user's mobile devices as data bridges between to unconnected intranets, keeping the stored data fully off the public internet.
\item[Secure transport]
	In the case where data is to be transported between two clients, this connection must always be as secure as technically possible.
	This removes the need to differentiate between utilizing public or private networks.
	The user could always rest assured that no eavesdropping can happen, even if the network is unsecured.
\item[Focus on ease of use]
	Secure software has the reputation of being cumbersome and hard to use correctly.
	By utilizing existing libraries we intend to make the software as easy to use as possible to allow a widespread adoption.
	This means that it must run by default upon installation after a simple setup process.
	Documentation and usability shall be one of the primary focuses.
	Technical details and complex settings shall be accessible but not required for extensive use.
\item[Encrypted third party capability]
	To enable full ease of use parity with third party services, peers shall be configurable to be encrypted data stores.
	This would mean that these peers retain a copy of the stored data but fully encrypted and unaccessible.
	Thus third party services may offer cloud storage while retaining the full trust of the user.
\end{description}

\subsection{Software Specifis}

% Golang
It is our intention to implement the core program in Golang~\cite{web:site:golang}.
The reason for this is primarily because Golang, or as more commonly known as Go, allows us to build concurrent software easily.
We plan on utilizing concurrency to make full use of parallel work to speed up all processing heavy work.
This will most likely include encryption and decryption of large data chunks, compression and decompression of the same to allow efficient network communications, and fast queries over large file directories.
Go is also a statically-typed language with garbage collection, type safety, and a large standard library: this removes a number of security risks from the start such as the infamous stack overflow attack.
Furthermore Go projects compile into a single executable file with all dependencies statically linked and included.
This decreases the perceived complexity of the software and allow easy portability.
Go also allows easy cross platform compilation, allowing simple support for a multitude of platforms.
Notably this includes ARM for mobile devices such as Android.

% tox
%TODO

\subsection{Thesis Scope}

The thesis is intended as an implementation work as a proof of concept that the proposed software set is implementable.
Therefore the goal is a working software suite.

To facilitate the implementation of the software we will start by planning the time frame and alloted work packages.
Then we will analyze existing solutions and highlight positive and negative aspects of comparable services and how they relate to our thesis.
Based on this we will plan out the exact feature set and required implementation work for our proposed software.
This includes the specification of existing work we will base our work on, namely the Tox protocol, and the API for system communication.
Then we will discuss the actual implementation.
Once done, we will analyze the results and compare them to the previously discussed comparable services.
Finally future improvements and extensions will be offered up in the hope that future work will be done to extend on our software.

\section{Structure of Proposal}

The words in parentheses are the suggested ones from the online text on proposals.

\begin{itemize}
\item Introduction (Summary)
\item Motivation (Hypothesis, problem or question) (Importance of topic)
\item Related Word (Prior research on topic)
\item ??? (Research approach or methodology)
\item Limitations and key assumptions
\item Contributions to knowledge
\item Descriptions of proposed chapters in dissertation
\end{itemize}

\bibliographystyle{alphadin} 	% letter + year sorted
{\small %smaller text means better URLs and compacter presentation
\bibliography{cite}
}

\end{document}
